\section{Ejercicio 20. *} \textit{Consideramos \(T:\mathbb{R}^3 \to
  \mathbb{R}^3\) una aplicación lineal, y la estructura de
  \(\mathbb{R}[X]\)-módulo correspondiente sobre \(\mathbb{R}^3\). Discutir los
  posibles valores de la longitud de \(\mathbb{R}^3\) como
  \(\mathbb{R}[X]\)-módulo, dependiendo de como sea \(T\). Poner un ejemplo de
  \(T\) para que se alcance cada longitud.}\\

Sabemos que un submódulo de nuestro \(\mathbb{R}[x]\)-módulo sobre
\(\mathbb{R}^3\) tiene  que ser un subespacio de este, por tanto las posibles
longitudes son \(2,3\) o \(4\).\\

Veamos que debe existir un subespacio invariante por \(T\) ya sea de dimensión 1
o dimensión 2.
Si tomamos un vector no nulo \(v \in \mathbb{R}^3\), sabemos que existen unos
coeficientes reales \(a_0, \dots, a_n\) tal que
\[
  0 = a_0v + \dots + a_nT^n(v) = (a_0I + \dots + a_nT^n)(v)
\]

Factoriamos este polinomio en \(T\) en factores irreducibles de grado 1 y 2.
Como el polinomio no es inyectivo, alguno de dichos factores debe no serlo.
\begin{itemize}
\item \textbf{Caso 1.} Un factor de grado 1 no es inyectivo, sea este factor de la
  forma \(T - \lambda I\), \(\exists u \in \mathbb{R}^3\) tal que \((T - \lambda
  I)(u) = 0 \implies T(u) = \lambda u \ \implies \ \langle u \rangle\) es
  invariante por \(T\). Notamos además que \(u\) es un vector propio.
\item \textbf{Caso 2.} Un factor de grado 2 no es inyectivo, sea ese factor
  \(T^2 + \alpha T + \beta I\), \(\exists u \in \mathbb{R}^3\) tal que \((T^2 +
  \alpha T + \beta I)(u)= 0\). Sea entonces \(U = <u,T(u)>\), como \(T^2(u) =
  -\alpha T(u) - \beta u \), \(U\) es un subespacio invariante bajo \(T\).
  Llamaremos a \(u, T(u)\) espacio propio por comodidad en la redacción.
\end{itemize}

Como estamos en \(\mathbb{R}^3\), la descomposición en irreducibles del
polinomio característico de \(T\) contiene un polinomio de grado 1, luego
siempre existe un vector propio.\\

Nos limitamos  a 3 casos en el estudio de la longitud (los casos
anteriores no son excluyentes), a saber
\begin{itemize}
\item Existe un valor propio \(u\) y no existe ningún espacio propio. Entonces
  tenemos la serie de composición
  \[
    0 \subset \langle u \rangle \subset \mathbb{R}
  \]
  luego la longitud es 3.
\item Existe un espacio propio \(\langle u, T(u) \rangle\), tal que \(u\) es
  un vector propio, tenemos entonces
  \[
    0 \subset \langle u \rangle \subset \langle u, T(u) \rangle\ \subset \mathbb{R}^3
  \]
\end{itemize}

En caso de tener \(2\) o \(3\) vectores propios, esta claro que tenemos al menos
un subespacio de dimensión \(1\) y otro de dimensión \(2\) invariantes por
\(T\), luego podríamos construir una serie de composición de longitud 4.\\

En caso de tener un único vector propio, solo existe un subespacio invariante,
luego podemos construir una serie de composición de longitud 3.\\


A continuación mostramos un ejemplo de aplicación lineal \(T:\mathbb{R}^3 \rightarrow \mathbb{R}^3\) que genere un módulo de cada una de las posibles longitudes.
Para dar la aplicación lineal únicamente tendremos que definir la imagen de una base de nuestro espacio, y por tanto utilizaremos la base usual
\(\{(1,0,0), (0,1,0), (0,0,1)\) por comodidad.\\

\textbf{Longitud 4}%
\[
\begin{aligned}
  T:\mathbb{R}^3 &\rightarrow \mathbb{R}^3\\
  x &\mapsto x
\end{aligned}
\]
Es sencillo comprobar que cualquier subespacio de \(\mathbb{R}^3\) es un
\(\mathbb{R}[X]\)-submódulo, luego tenemos la siguiente cadena
\[
        0 \subset \langle (1,0,0) \rangle \subset \langle (1,0,0), (0,1,0)
        \rangle \subset \mathbb{R}^3
\]

\textbf{Longitud 3}%
\[
\begin{aligned}
  T:\mathbb{R}^3 &\rightarrow \mathbb{R}^3\\
  (1,0,0) &\mapsto (0,1,0)\\
  (0,1,0) &\mapsto (0,0,1)\\
  (0,0,1) &\mapsto (1,0,0)
\end{aligned}
\]

En este caso, vemos que que el único subespacio de dimensión 1 que es un
submódulo, resulta \(\langle (1,1,1) \rangle\), tomemos ahora un vector
linealmente independiente \(\alpha\).

\[
  \begin{aligned}
  T(a(1,1,1) + b\alpha) &= a(1,1,1) + bT(\alpha_1(1,0,0) + \alpha_2(0,1,0) +
  \alpha_3(0,0,1))\\
  &= a(1,1,1) + b(\alpha_3, \alpha_1, \alpha_2) \not \in
  \langle(1,1,1),\alpha\rangle
  \end{aligned}
\]

Luego \(\langle (1,1,1)\rangle\) es el único submodulo que tenemos

\[
        0 \subset \langle (1,1,1) \rangle \subset \mathbb{R}^3
\]
