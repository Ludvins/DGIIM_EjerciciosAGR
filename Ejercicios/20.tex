\section{Ejercicio 20. *} \textit{Consideramos \(T:\mathbb{R}^3 \to
  \mathbb{R}^3\) una aplicación lineal, y la estructura de
  \(\mathbb{R}[X]\)-módulo correspondiente sobre \(\mathbb{R}^3\). Discutir los
  posibles valores de la longitud de \(\mathbb{R}^3\) como
  \(\mathbb{R}[X]\)-módulo, dependiendo de como sea \(T\). Poner un ejemplo de
  \(T\) para que se alcance cada longitud.}\\

Sabemos que un submódulo de nuestro \(\mathbb{R}[x]\)-módulo sobre
\(\mathbb{R}^3\) tiene  que ser un subespacio de este, por tanto las posibles
longitudes son \(2,3\) o \(4\).\\

Veamos que, para cualquier espacio real de dimensión \(n\), y cualquier operador
lineal \(T\), debe existir un subespacio invariante por \(T\) ya sea de
dimensión 1 o dimensión 2, donde un subespacio invariante es equivalente a ser
un submódulo, esto nos permitirá descartar el caso en el que no exista ningún
submódulo y la longitud sea 2.\\

Si tomamos un vector no nulo \(v \in \mathbb{R}^n\), sabemos que existen unos
coeficientes reales \(a_0, \dots, a_n\) tal que
\[
  0 = a_0v + \dots + a_nT^n(v) = (a_0I + \dots + a_nT^n)(v)
\]
Factoriamos este polinomio en \(T\) en factores irreducibles de grado 1 y 2.
Como el polinomio no es inyectivo, alguno de dichos factores debe no serlo.
\begin{itemize}
\item \textbf{Caso 1.} Un factor de grado 1 no es inyectivo, sea este factor de la
  forma \(T - \lambda I\), \(\exists u \in \mathbb{R}^3\) tal que \((T - \lambda
  I)(u) = 0 \implies T(u) = \lambda u \ \implies \ \langle u \rangle\) es un submódulo. Notamos además que \(u\) es un vector propio.
\item \textbf{Caso 2.} Un factor de grado 2 no es inyectivo, sea ese factor
  \(T^2 + \alpha T + \beta I\), \(\exists u \in \mathbb{R}^3\) tal que \((T^2 +
  \alpha T + \beta I)(u)= 0\). Sea entonces \(U = <u,T(u)>\), como \(T^2(u) =
  -\alpha T(u) - \beta u \), \(U\) es un submódulo.
\end{itemize}
Notamos que como polinomio podemos utilizar el polinomio característico de la
aplicación.\\

Además es facil comprobar que todos los submódulos se generan de
esta forma, pues sea \(\langle u \rangle = U\) un submódulo, se tiene que \(T(u)
\in U \implies \exists a \in \mathbb{R}\) tal que \(T(u) = au\), luego \(u\) es un
vector propio. En caso de ser \(U = \langle u , v \rangle\), si alguno de ellos
no es un vector propio, digamos por ejemplo \(u\), entonces \(T(u)\) y \(u\) son linealmente independientes y
\(T(u) \in U \implies U = \langle u , T(u) \rangle\).\\

Como estamos en \(\mathbb{R}^3\), la descomposición en irreducibles del
polinomio característico de \(T\) contiene un polinomio de grado 1, luego
siempre existe un vector propio.\\

Nos limitamos a tres casos en el estudio de la longitud (los casos
anteriores no son excluyentes), a saber
\begin{itemize}
\item Existe un único valor propio, y para el vector propio asociado \(u\)  no existe ningún espacio de la forma
  \(\langle v, T(v)\rangle \) que contenga a \(u\), invariante por \(T\). Entonces tenemos la serie de composición
  \[
    0 \subset \langle u \rangle \subset \mathbb{R}^3
  \]
  luego la longitud es 3.
\item Existe un único valor propio, y el vector propio asociado \(u\) se encuentra dentro de un submódulo de
  dimensión 2 \(\langle v, T(v) \rangle\), tenemos entonces
  \[
    0 \subset \langle u \rangle \subset \langle v, T(v) \rangle\ \subset \mathbb{R}^3
  \]
  luego la longitud es 4.
\item Existen 3 valores propios, en cuyo caso, tomando dos vectores propios \(u,v\), tenemos que
\[
0 \subset \langle u \rangle \subset \langle u, v \rangle \subset \mathbb{R}^3
\]
luego la longitud es 4.
\end{itemize}

A continuación mostramos un ejemplo de aplicación lineal \(T:\mathbb{R}^3 \rightarrow \mathbb{R}^3\) que genere un módulo de cada una de las posibles longitudes.
Para dar la aplicación lineal únicamente tendremos que definir la imagen de una base de nuestro espacio, y por tanto utilizaremos la base usual
\(\{(1,0,0), (0,1,0), (0,0,1)\) por comodidad.\\

\textbf{Longitud 4}%
\[
\begin{aligned}
  T:\mathbb{R}^3 &\rightarrow \mathbb{R}^3\\
  x &\mapsto x
\end{aligned}
\]

En este caso, todos los vectores son vectores propios luego es sencillo comprobar que cualquier subespacio de \(\mathbb{R}^3\) es un
\(\mathbb{R}[X]\)-submódulo, la siguiente serie de composición nos da la longitud buscada.
\[
        0 \subset \langle (1,0,0) \rangle \subset \langle (1,0,0), (0,1,0)
        \rangle \subset \mathbb{R}^3
\]

\textbf{Longitud 3}%
\[
\begin{aligned}
  T:\mathbb{R}^3 &\rightarrow \mathbb{R}^3\\
  (1,0,0) &\mapsto (0,1,0)\\
  (0,1,0) &\mapsto (0,0,1)\\
  (0,0,1) &\mapsto (1,0,0)
\end{aligned}
\]

En este caso se puede calcular que el polinomio característico de la aplicación
es \(-x^3 + 1 = -(x-1)(x^2+x+1)\), luego tenemos un único vector propio, \(u = (1,1,1)\). Además, \(T^2 + T +Id = 3Id\) es inyectiva, luego no existe ningún submodulo de dimensión 2.

\[
        0 \subset \langle (1,1,1) \rangle \subset \mathbb{R}^3
\]
