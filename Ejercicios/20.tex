\section{Ejercicio 20. *} \textit{Consideramos \(T:\mathbb{R}^3 \to
  \mathbb{R}^3\) una aplicación lineal, y la estructura de
  \(\mathbb{R}[X]\)-módulo correspondiente sobre \(\mathbb{R}^3\). Discutir los
  posibles valores de la longitud de \(\mathbb{R}^3\) como
  \(\mathbb{R}[X]\)-módulo, dependiendo de como sea \(T\). Poner un ejemplo de
  \(T\) para que se alcance cada longitud.}\\

Sabemos que un submódulo de nuestro \(\mathbb{R}[x]\)-módulo sobre \(mathbb{R}^3\) tiene  que ser un subespacio de este, por tanto las posibles longitudes son:
\begin{itemize}
    \item 2. Supuesto que no podemos encontrar ningún submódulo distinto del total o el vacío.
    \item 3. En cuyo caso, el subespacio asociado al submódulo podrá ser de dimensión 1 o 2.
    \item 4. En cuyo caso, tendremos un subespacio de dimensión 1, y otro de dimensión 2.
\end{itemize}

A continuación mostramos un ejemplo de aplicación lineal \(T:\mathbb{R}^3 \rightarrow \mathbb{R}^3\) que genere un módulo de cada una de las posibles longitudes.
Para dar la aplicación lineal únicamente tendremos que definir la imagen de una base de nuestro espacio, y por tanto utilizaremos la base usual
\(\{(1,0,0), (0,1,0), (0,0,1)\) por comodidad.\\

\textbf{Longitud 2}%

ESTO ESTÁ MAL
\[
T:\mathbb{R}^3: \rightarrow \mathbb{R}^3
\]
tal que
\[
T((1,0,0)) = (0,1,0),
\]
\[
T((0,1,0)) = (0,0,1),
\]
\[
T((0,0,1)) = (1,0,0)
.\]
Es facil ver que la dimesión de cualquier submódulo distinto del \({0}\) tiene que ser el total, pues en cuanto tengas un vector distinto del vacío, aplicando \(T\) una y dos veces obtiene.

\textbf{Longitud 3}
\[
T:\mathbb{R}^3: \rightarrow \mathbb{R}^3
\]
tal que
\[
T((1,0,0)) = (1,0,0),
\]
\[
T((0,1,0)) = (0,0,1),
\]
\[
T((0,0,1)) = (0,1,0)
.\]

Con esto, la cadena de longitud 3 sería:
\[
{0} \subset <(0,1,0), ()
.\]
