\section{Ejercicio 20. *} \textit{Consideramos \(T:\mathbb{R}^3 \to
  \mathbb{R}^3\) una aplicación lineal, y la estructura de
  \(\mathbb{R}[X]\)-módulo correspondiente sobre \(\mathbb{R}^3\). Discutir los
  posibles valores de la longitud de \(\mathbb{R}^3\) como
  \(\mathbb{R}[X]\)-módulo, dependiendo de como sea \(T\). Poner un ejemplo de
  \(T\) para que se alcance cada longitud.}\\

Notemos primero que la noción de submódulo equivale a subgrupo cerrado bajo la
acción del anillo, y los únicos subgrupos que tenemos son

\[
   \langle (1,0,0) \rangle \hspace{1cm}  \langle (0,1,0) \rangle \hspace{1cm} \langle (0,0,1) \rangle
\]
\[
  \langle (1,0,0),(0,1,0) \rangle \hspace{1cm}\langle (1,0,0),(0,0,1) \rangle \hspace{1cm}\langle (0,0,1),(0,1,0) \rangle
\]


Por ello, la mayor cadena de subgrupos maximales que vamos a
poder obtener será de longitud 3, por ejemplo,

\[
  0 \subset \langle (1,0,0) \rangle \subset \langle (1,0,0),(0,1,0) \rangle
  \subset \mathbb{R}^3
\]
que sean submódulos o no dependerá de \(T\). Por ejemplo, si tomamos \(T = id\),
entonces estamos ante una serie de composición de \(\mathbb{R}^3\) como
\(\mathbb{R}[X]\)-módulo.

Pero si tomamos \(T\) tal que la imagen de cualquiera de los generadores sea
\((1,0,0)\), entonces el único submódulo que tenemos es \(\langle (1,0,0)
\rangle\), luego tendríamos la serie de composición
\[
  0 \subset \langle (1,0,0) \rangle \subset \mathbb{R}^3
\]

lo que nos daría una longitud de 2.

Por último, si hacemos que \(T\) permite los elementos de la base, entonces no
habría ningun submódulo propio y tendríamos una longitud de 1.
