\section{Ejercicio 20. *} \textit{Consideramos \(T:\mathbb{R}^3 \to
  \mathbb{R}^3\) una aplicación lineal, y la estructura de
  \(\mathbb{R}[X]\)-módulo correspondiente sobre \(\mathbb{R}^3\). Discutir los
  posibles valores de la longitud de \(\mathbb{R}^3\) como
  \(\mathbb{R}[X]\)-módulo, dependiendo de como sea \(T\). Poner un ejemplo de
  \(T\) para que se alcance cada longitud.}\\

Sabemos que un submódulo de nuestro \(\mathbb{R}[x]\)-módulo sobre
\(\mathbb{R}^3\) tiene  que ser un subespacio de este, por tanto las posibles
longitudes son \(2,3\) o \(4\).\\

Como estamos tatando con \(\mathbb{R}\) un cuerpo algebraicamente cerrado,
sabemos que \(T\) tiene al menos un vector propio. Estos vectores propios son
los que nos marcarán la longitud de \(\mathbb{R}^3\), ya que generan subespacios
vectoriales invariantes por \(T\), es decir, submódulos.\\

En caso de tener \(2\) o \(3\) vectores propios, esta claro que tenemos al menos
un subespacio de dimensión \(1\) y otro de dimensión \(2\) invariantes por
\(T\), luego podríamos construir una serie de composición de longitud 4.\\

En caso de tener un único vector propio, solo existe un subespacio invariante,
luego podemos construir una serie de composición de longitud 3.\\


A continuación mostramos un ejemplo de aplicación lineal \(T:\mathbb{R}^3 \rightarrow \mathbb{R}^3\) que genere un módulo de cada una de las posibles longitudes.
Para dar la aplicación lineal únicamente tendremos que definir la imagen de una base de nuestro espacio, y por tanto utilizaremos la base usual
\(\{(1,0,0), (0,1,0), (0,0,1)\) por comodidad.\\

\textbf{Longitud 4}%
\[
\begin{aligned}
  T:\mathbb{R}^3 &\rightarrow \mathbb{R}^3\\
  x &\mapsto x
\end{aligned}
\]
Es sencillo comprobar que cualquier subespacio de \(\mathbb{R}^3\) es un
\(\mathbb{R}[X]\)-submódulo, luego tenemos la siguiente cadena
\[
        0 \subset \langle (1,0,0) \rangle \subset \langle (1,0,0), (0,1,0)
        \rangle \subset \mathbb{R}^3
\]

\textbf{Longitud 3}%
\[
\begin{aligned}
  T:\mathbb{R}^3 &\rightarrow \mathbb{R}^3\\
  (1,0,0) &\mapsto (0,1,0)\\
  (0,1,0) &\mapsto (0,0,1)\\
  (0,0,1) &\mapsto (1,0,0)
\end{aligned}
\]

En este caso, vemos que que el único subespacio de dimensión 1 que es un
submódulo, resulta \(\langle (1,1,1) \rangle\), tomemos ahora un vector
linealmente independiente \(\alpha\).

\[
  \begin{aligned}
  T(a(1,1,1) + b\alpha) &= a(1,1,1) + bT(\alpha_1(1,0,0) + \alpha_2(0,1,0) +
  \alpha_3(0,0,1))\\
  &= a(1,1,1) + b(\alpha_3, \alpha_1, \alpha_2) \not \in
  \langle(1,1,1),\alpha\rangle
  \end{aligned}
\]

Luego \(\langle (1,1,1)\rangle\) es el único submodulo que tenemos

\[
        0 \subset \langle (1,1,1) \rangle \subset \mathbb{R}^3
\]
