\section{Ejercicio 12.}%
\label{sec:ejercicio_12_}

\textit{Demostrar que un conjunto de generadores $m_i: i \in I$ de un módulo  ${}_AM$ es una base si, y solo si, la igualdad $\sum_i r_i m_i = 0$ para $r_i \in A$ implica $r_i = 0$ para todo $\forall i \in I$. Dar un ejemplo de módulo no nulo finitamente generado que no sea libre.}\\

Razonemos por contradicción para la primera implicación. Supongamos que $m_1, \dots, m_n$ es base de ${}_AM$ y que existen $r_1, \dots, r_n$, con $r_k \neq 0$ tal que $\sum_i r_i m_i = 0$. Sea $m \in M$ con $m = \sum_i a_i m_i$. Entonces
\[
m = \sum_i a_i m_i = \sum_i a_i m_i + \sum_i r_i m_i = \sum_i(a_i+r_i)m_i
\]
con $a_k + r_k \neq a_k$. Por tanto $m_1, \dots, m_n$ no sería base.

Para la otra implicación, supongamos que existen $ a_1, \dots, a_n, a'_1, \dots a'_n \in A$ tales que $\sum_i a_im_i= \sum_i a'_im_i$. Entonces,
\[
\sum_i a_i m_i - \sum_i a'_i m_i = 0 \implies \sum_i (a_i - a'_i)m_i = 0 \implies a_i = a'_i \quad \forall i \in I
.\]

Un ejemplo de módulo no nulo finitamente generado que no sea libre es $\mathbb{Z}_2$ visto como $\mathbb{Z}$-módulo. Claramente es finitamente generado pues solo tiene 2 elementos, y la única posible base sería ${1}$, pero no lo es por $2 \cdot 1 = 0$
