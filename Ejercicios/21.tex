\section{Ejercicio 21. } \textit{Sea \(\mathbb{P}_n\) el espacio vectorial real
  de las funcione polinómicas en una variable de grado menor o igual que \(n\).
  Sea \(T: \mathbb{P}_n \to \mathbb{P}_n\) la aplicación lineal que asigna a
  cada polinomio su derivada. Calcular una serie de composición de
  \(\mathbb{P}_n\) visto como \(\mathbb{R}[X]\)-módulo via \(T\)}.\\

Consideremos los espacios vectoriales \(\mathbb{P}_{n-1},\dots,\mathbb{P}_0\),
es claro que \(\mathbb{P}_i \subset \mathbb{P}_{i+1}\) es un subgrupo y es
cerrado bajo derivación, luego es un submódulo. Esto nos permite crear la cadena

\[
  0 \subset \mathbb{P}_0 \subset \dots \subset \mathbb{P}_n
\]

Solo nos queda comprobar que cada eslabón este formado por un submódulo maximal,
tomamos \(\mathbb{P}_i \subset \mathbb{P}_{i+1}\),  y añadamos un polinomio \(p\) de
grado \(i+1\) a \(\mathbb{P}_i \), entonces es claro que con las operaciones de
grupo podemos construir cualquier polinomio de grado \(i+1\), luego \(\langle
(p, \mathbb{P}_i) \rangle = \mathbb{P}_{i+1}\). Por lo tanto, los eslabones son
maximales y tenemos una serie de composición.
