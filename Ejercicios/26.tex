\section{Ejercicio 26. **} \textit{En las condiciones del ejercicio anterior,
  demostrar que el polinomio característico de \(T\) es \(m(X)^t\)}.\\

Comencemos viendo que si \(f = gh\), son 3 polinomios donde \(g,h\) son
coprimos, entonces
\[
  Ker(f(T)) = Ker(g(T)) \oplus Ker(h(T))
\]
Comencemos notando que \(Ker(g(T)), Ker(h(T)) \in Ker(f(T))\) y \(Ker(g(T)) +
Ker(h(T)) \in Ker(f(T))\).\\ Por ser coprimos, existen polinomios \(u,v\) tales
que
\[
  1 = gu + hv \implies Id = u(T)g(T) + v(T)h(T)
\]
Tomemos entonces \(\alpha \in Ker(f(T))\), tenemos \(\alpha = u(T)g(T)\alpha +
v(T)h(T)\alpha\).
Donde  \( u(T)g(T)\alpha \in Ker(h(T))\) y  \( v(T)h(T)\alpha \in Ker(g(T))\),
luego concluimos que \(Ker(f(T)) = Ker(g(T)) + Ker(h(T))\).
