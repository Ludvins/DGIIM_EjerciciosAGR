\section{Ejercicio 26. **} \textit{En las condiciones del ejercicio anterior,
  demostrar que el polinomio característico de \(T\) es \(m(X)^t\)}.\\

Probemos la siguiente proposición

\textbf{Proposición. }Sean \(f,g,h \in K[X]\) tales que \(f = gh\), y \(g,h\) son
coprimos, entonces
\[
  Ker(f(T)) = Ker(g(T)) \oplus Ker(h(T))
\]
\textbf{Demostración.}\\
Comencemos notando que \(Ker(g(T)), Ker(h(T)) \in Ker(f(T))\) y \(Ker(g(T)) +
Ker(h(T)) \in Ker(f(T))\).\\ Por ser coprimos, existen polinomios \(u,v\) tales
que
\[
  1 = gu + hv \implies Id = u(T)g(T) + v(T)h(T)
\]
Tomemos entonces \(\alpha \in Ker(f(T))\), tenemos \(\alpha = u(T)g(T)\alpha +
v(T)h(T)\alpha\).
Donde  \( u(T)g(T)\alpha \in Ker(h(T))\) y  \( v(T)h(T)\alpha \in Ker(g(T))\),
luego concluimos que \(Ker(f(T)) = Ker(g(T)) + Ker(h(T))\).

Supongamos que existe \(\beta\) tal que \(g(T)\beta = h(T)\beta = 0\), luego
tenemos que
\[
  0 = u(T)g(T)\beta + v(T)h(T)\beta = \beta
\]

\textbf{Proposición.} Sea \(T: V \rightarrow V \) un endomorfismo \(K\)-lineal, con \(V\) espacio vectorial de dimensión finita. Entonces, si el polinomio mínimo \(m(X)\) de  \(T\) es irreducible en  \(K[X]\), el polinomio característico \(p(X)\) de T se escribe como \(m(X)^t\) para algún \(t \in \mathbb{N}\).\\

\textbf{Demostración.}\\
Razonando por contradicción, supongamos que el polinomio característico de \(T\) es  \(mg\), donde  \(g\) es primo relativo con \(m\). Entonces, usando la proposición anterior tenemos que \(V = Ker(m(T)) \oplus Ker(g(T))\). Es facil comprobar que el polinomio mínimo de \(T\) es el mínimo común múltiplo del polinomio mínimo de \(T\restriction_{Ker(m(T))}\) y el polinomio mínimo \(T\restriction_{Ker(g(T))}\).\\

Ahora, como el polinomio mínimo de \(T\restriction_{Ker(m(T))}\) es \(m\) también, y este es irreducible, tenemos que \(g = 1\), lo que prueba nuestro resultado.
