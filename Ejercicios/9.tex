\section{Ejercicio 8.} \emph{Sea}
\[
  \mathbb{H} = \Big\{ \begin{bmatrix}
    \alpha       &  -\bar{\beta} \\
    \beta       &  \bar{\alpha}
\end{bmatrix} \ : \ \alpha , \beta \in \mathbb{C} \Big\}
\]
\emph{1. Demostrar que \(\mathbb{H}\) es una subálgebra real de
  \(M_2(\mathbb{C})\) y que \(Z(\mathbb{H}) = \mathbb{R}\)}\\
\emph{2. Demostrar que todo elemento no nulo de \(\mathbb{H}\) es una unidad}\\
\emph{3. Demostrar que las matrices}

\[
  \text{\textbf{1}}  = \begin{bmatrix}
    1  &  0 \\
    0  &  1
\end{bmatrix},   \text{\textbf{i}} = \begin{bmatrix}
    0  &  -1 \\
    1  &  0
\end{bmatrix},  \text{\textbf{j}}  = \begin{bmatrix}
    i  &  0 \\
    0  &  -i
\end{bmatrix},   \text{\textbf{k}}  = \begin{bmatrix}
    0  &  i \\
    i  &  0
\end{bmatrix}
\]
\emph{forman una base de \(\mathbb{H}\) como espacio vectorial real.}\\
\emph{4. Comprobar las identidades}
\[
  \text{\textbf{i}} ^2 = \text{\textbf{j}} ^2 = \text{\textbf{k}} ^2 = -\text{\textbf{1}} , \
  \text{\textbf{i}} \text{\textbf{j}}  = \text{\textbf{k}} , \ \text{\textbf{j}} \text{\textbf{k}}  = \text{\textbf{i}} , \ \text{\textbf{k}} \text{\textbf{i}}  = \text{\textbf{j}}
\]

Para ver que es una subálgebra, vemos que \(\mathbb{H}\) es un subespacio
vectorial de \(M_2(\mathbb{C})\), vemos que es cerrado para la suma de matrices

\[
  \begin{bmatrix}
    \alpha_1       &  -\bar{\beta_1} \\
    \beta_1       &  \bar{\alpha_1}
\end{bmatrix} \begin{bmatrix}
    \alpha_2       &  -\bar{\beta_2} \\
    \beta_2       &  \bar{\alpha_2}
\end{bmatrix} = \begin{bmatrix}
    \alpha_1 + \alpha_2 & -\bar{\beta_1} + \bar{\beta_2} \\
    \beta_1 + \beta_2 & \bar{\alpha_1} +\bar{\alpha_2}
\end{bmatrix}
\]

y para la multiplicación

\[
  \begin{bmatrix}
    \alpha_1       &  -\bar{\beta_1} \\
    \beta_1       &  \bar{\alpha_1}
\end{bmatrix} \begin{bmatrix}
    \alpha_2       &  -\bar{\beta_2} \\
    \beta_2       &  \bar{\alpha_2}
\end{bmatrix} = \begin{bmatrix}
    \alpha_1\alpha_2 - \bar{\beta_1}\beta_2 & - \alpha_1 \bar{\beta_2} - \bar{\beta_1}\bar{\alpha_2} \\
    \beta_1\alpha_2 + \bar{\alpha_1}\beta_2       &  -\beta_1\bar{\beta_2} + \bar{\alpha_1}\bar{\alpha_2}
\end{bmatrix}
\]
además \(1 \in \mathbb{H}\)

Para que un elemento esté en el centro deben coincidir

\[
  \begin{bmatrix}
    \alpha_1       &  -\bar{\beta_1} \\
    \beta_1       &  \bar{\alpha_1}
\end{bmatrix} \begin{bmatrix}
    \alpha_2       &  -\bar{\beta_2} \\
    \beta_2       &  \bar{\alpha_2}
\end{bmatrix} = \begin{bmatrix}
    \alpha_1\alpha_2 - \bar{\beta_1}\beta_2 & - \alpha_1 \bar{\beta_2} - \bar{\beta_1}\bar{\alpha_2} \\
    \beta_1\alpha_2 + \bar{\alpha_1}\beta_2       &  -\beta_1\bar{\beta_2} +
    \bar{\alpha_1}\bar{\alpha_2}
\end{bmatrix}
\]
\[
   \begin{bmatrix}
    \alpha_2       &  -\bar{\beta_2} \\
    \beta_2       &  \bar{\alpha_2}
\end{bmatrix} \begin{bmatrix}
    \alpha_1       &  -\bar{\beta_1} \\
    \beta_1       &  \bar{\alpha_1}
\end{bmatrix}= \begin{bmatrix}
    \alpha_2\alpha_1 - \bar{\beta_2}\beta_1 & - \alpha_2 \bar{\beta_1} - \bar{\beta_2}\bar{\alpha_1} \\
    \beta_2\alpha_1 + \bar{\alpha_2}\beta_1       &  -\beta_2\bar{\beta_1} + \bar{\alpha_2}\bar{\alpha_1}
\end{bmatrix}
\]
Para tener esto necesitamos \(\beta_1 \alpha_2 + \bar{\alpha_1}\beta_2 =
\beta_2\alpha_1 + \bar{\alpha_2}\beta_1  \ \implies \ \beta_1 = 0\) y \(\alpha_1
= \bar{\alpha_1}\). Luego \(\alpha \in \mathbb{R}\)
\[
  Z(\mathbb{H}) =  \Big\{\begin{bmatrix}
    a       &  0 \\
   0 &  a
 \end{bmatrix}  : \ a \in \mathbb{R} \Big\} \cong \mathbb{R}
\]


2. Para ver que todo elemento es una unidad basta tomar
\[
   \begin{bmatrix}
    \alpha       &  -\bar{\beta} \\
    \beta       &  \bar{\alpha}
  \end{bmatrix} ^{-1} =
   \begin{bmatrix}
    \bar{\alpha}/\|\alpha\|       &  \bar{\beta}/\|\beta\| \\
    -\beta/\|\beta\|       &  \alpha / \|\alpha\|
\end{bmatrix}
\]

3. Veamos ahora que dichas matrices son una base, es sencillo ver que son
linealmente independientes, luego comprobemos que son un sistema de generadores

\[
   \begin{bmatrix}
    \alpha       &  -\bar{\beta} \\
    \beta       &  \bar{\alpha}
  \end{bmatrix}  = Re(\alpha)\text{\textbf{1}} + Re(\beta)\text{\textbf{i}} + Im(\alpha)\text{\textbf{j}} + Im(\beta)\text{\textbf{k}}
\]

4. Para comprobar dichas identidades basta con realizar las cuentas correspondientes.
