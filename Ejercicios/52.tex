\section{Ejercicio 52.} \emph{ Deducir cuantas representaciones irreducible complejas no equivalentes tiene \(S_{3}\), y cuales son sus dimensiones. }\\

Si tomamos la representación definida en el ejercicio 48 y la redefinimos sobre los complejos, tenemos una representación irreducible compleja de dimensión 2.

Ahora como \(|S_{3}| = 6 = 4+1+1\),  existen otras dos representaciones complejas irreducibles. comenzamos viendo que cualquier representación de dimensión 1 es irreducible pues no existen subespacios propios.

Como representaciones tomamos, la representación trivial, que lleva cada elemento a la aplicación identidad, y la representación que lleva a \((1\ 2\ 3)\) y \((1 \ 2)\) a la aplicación multiplicar por \(i\) (la imagen del resto de elementos está totalmente determinada).
