\section{Ejercicio 56.}%
\label{sec:ejercicio_56_}

\textit{Demostrar que todo carácter de un grupo finito \(G\) es constante sobre cada clase de conjugación de  \(G\)}\\

Sea \(\chi = \chi_{\rho}\) para una representación \(\rho\) de \(G\), \(g \in G\) y \(h = aga^{-1}\) para algún \(a \in G\). Entonces,
\[
\chi(h) = \operatorname{tr} (\rho(aga^{-1})) = \operatorname{tr} (\rho(a)\rho(g)\rho(a)^{-1}) = \operatorname{tr}(\rho(g)) = \chi(g)
.\]

Donde hemos utilizado que \(\rho\) es un morfismo de grupos y que la traza es invariante bajo semejanza.
