\section{Ejercicio 22. **} \textit{En las condiciones del ejercicio anterior, calcular
  todos los \(\mathbb{R}[X]\)-submódulos de \(\mathbb{P}_n\).}\\

Comencemos viendo que los subespacios vectoriales \(\mathbb{P}_i\) con \(i
\leq n\), son también \(\mathbb{R}[X]\)-submódulos. Esto es evidente pues la
aplicación \(T\) consiste es la derivación. Luego estos espacios son cerrados
ante \(T\).\\

Comprobemos ahora que estos son los únicos submódulos que tenemos. Para ello
tomemos un polinomio \(p\) cualquiera de grado \(m < n\). Vamos a ver quien es
el submódulo que genera \(\langle p \rangle\).\\

Supongamos que
\[
  p = a_0 + a_1x + \dots + a_m x^m
\]

Consideremos entonces \(T^m(p) = m!a_m \in \mathbb{R}\), luego \(1 \in \langle p
\rangle\). De la misma forma,
\[T^{m-1}(p) = (m-1)!a_{m-1} + m!a_mx \implies
\frac{T^{m-1}(p) - (m-1)!a_{m-1}}{m!a_m} = x \in \langle p \rangle\]

Siguiendo un proceso inductivo, podemos ver entonces que \(x^i \in \langle p
\rangle \ \forall i = 0,\dots,n\). Luego \(\langle p \rangle = \mathbb{P}_n\).\\

Por ello, concluimos que \(\{\mathbb{P}_i\}_{i\leq n}\) son todos los
\(\mathbb{R}[X]\)-submódulos de \(\mathbb{P}_n\).
