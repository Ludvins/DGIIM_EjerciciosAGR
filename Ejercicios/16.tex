\section{Ejercicio 16}%
\label{sec:ejercicio_16}
\textit{Siguiendo la notación del Ejercicio 15, ¿para qué valores \(\theta, \theta'\) son los \(\mathbb{R}[X]\)-módulos \(V_\theta, V_{\theta'}\) isomorfos?}

Idea: Si es \(q \cdot \pi\) siendo \(q\) racional únicamente si \(\theta y \theta'\) son el mismo ángulo u opuestos. Si son el mismo el morfismo es la identidad, y si son opuestos, fijas un vector, y el morfismo es el que lleva cada vector en el simétrico respecto del eje dado por dicho vector. En caso contrario, tras aplicar un número de veces el giro sobre \(V_\theta\) volvemos al vector original, y sin embargo eso no ocurre en el nuevo. (No está formalizado y es muy probable que esté mal.)
