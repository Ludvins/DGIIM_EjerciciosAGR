\section{Ejercicio 62}%
\label{sec:ejercicio_62}

\textit{Sea \(G\) un grupo abeliano finito, y sea \(\widehat{G}\) el conjunto de los caracteres complejos irreducibles de \(G\). Demostrar que el producto inducido por el de números complejos dota a \(\widehat{G}\) de estructura de grupo.} \\

En primer lugar mencionemos que por ser \(G\) un grupo abeliano finito, sabemos que el número de caracteres complejos irreducibles coincide con el orden del grupo, y por tanto \(\chi(1) = 1\) para todo  \(\chi \in \widehat{G}\). Además, sabemos que cualquier caracter complejo de orden 1 es irreducible, pues no puede tener subespacios propios.  \\

Ahora, dados \(\chi, \chi' \in \widehat{G}\), definimos \(\rho: G \to GL(\mathbb{C})\) con
\[
\rho(g) = \chi(g) \chi'(g) \text{ para todo } g \in G
.\]

Si demostramos que \(\rho\) es una representación compleja de \(G\) habríamos terminado, pues al ser de orden uno, el caracter complejo asociado \(\chi_{\rho}\) es irreducible y se cumple \(\chi_{\rho}(g) = \chi(g) \chi'(g) \in \widehat{G}\) para todo \(g \in G\). \\

Comprobemos entonces que \(\rho\) es un morfismo para concluir la demostración.  Por tanto, \(\rho(1) = \chi(1)\chi'(1) = 1\). Por otro lado, dados \(g, h \in G\), como ya dijimos que todos los caracteres irreducibles son de orden 1, tenemos que
\[
\rho(gh) = \chi(gh) \chi'(gh) = \chi(g)\chi(h)\chi'(g)\chi'(h) = \rho(g)\rho(h)
.\]

Luego hemos probado que \(\rho\) es un morfismo, concluyendo así la demostración.
