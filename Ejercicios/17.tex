\section{Ejercicio 17}%
\label{sec:ejercicio_17}
\textit{Sea \(M\) un \(A\)-módulo. Demostrar que \(M\) es simple si, y solo si, \(M = Am\) paa todo \(0\neq m \in M\)}.\\

Claramente, si \(M\) es simple, por \(Am\) submódulo tiene que ser \(M\) o  \(\{0\}\), y por  \(m\neq 0\) tenemos que  \(Am = M\).

La otra implicación es también casi inmediata. Supongamos que existe \(N \subset M\) submódulo distinto de \(\{0\}\). Entonces, sea \(n \in N\) distinto de  \(0\), tenemos que  \(An \subset N\), pero como  \(An = M\),  \(N = M\).
