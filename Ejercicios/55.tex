\section{Ejercicio 55.}
\textit{Demostrar que, para todo \(a \in \mathbb{C}G\)}, se tiene que \(\tilde{\chi}_{\rho}(a) = \operatorname{tr} \tilde{\rho} (a)\). \\

Sea \(a = \sum_{g \in G} \lambda_g g \) con \(\lambda_g \in \mathbb{C}\) para todo \(g \in G\). Entonces,
\[
\tilde{\chi}_{\rho}(a) = \sum_{g \in G} \lambda_g \chi_{\rho}(g) = \sum_{g \in G} \lambda_g \operatorname{tr}(\rho(g)) = \operatorname{tr} (\sum_{g \in G} \lambda_g \rho (g)) = \operatorname{tr} (\tilde{\rho}(a))
,\]

donde hemos solo hemos utilizado la definición de \(\tilde{\chi}_{\rho}\) y de \(\tilde{\rho}\), y la linealidad de la traza.
