
\section{Ejercicio 5.} \emph{Sea \(A\) un espacio vectorial sobre un cuerpo
  \(K\). Demostrar que dar una estructura de \(K\)-álgebra asociativa unital
  sobre \(A\) es equivalente a dar una multiplicación asociativa \(K\)-bilineal \(\star :A
  \times A \to A\) junto con una aplicación \(K\)-lineal \(\tau : K \to A\) tal
  que \(\tau(k)\star a = ka = a \star \tau(k) \ \forall k \in K, a \in A\)}\\

Supongamos que tenemos una estructura de \(K\)-álgebra sobre \(A\). Denotamos \(\star\) a la
multiplicación de \(A\) como anillo y \(\tau :K \to Z(A)\) al morfismo que dota
de estructura de \(K\)-álgebra.
 Veamos que \(\tau\) es \(K\)-lineal, sea \(k \in K\):
\[
\tau(k) = \tau(k) \star 1_A = k1_A = k \tau(1_K)
\]
Comprobemos ahora que \(\star\) es \(K\)-bilineal, la bilinealidad viene dada
por la estructura de anillo. Sean \(k \in K, a,b \in A\)
\[
k(a \star b) = \tau(k) \star (a \star b) = (\tau(k) \star a) \star b = (ka)
\star b
\]
\[
k(a \star b) = \tau(k) \star (a \star b) = (\tau(k) \star a) \star b = (a \star
\tau(k)) \star b = a \star (\tau(k) \star b) = a \star (kb)
\]

Supongamos ahora que tenemos ambas aplicaciones definidas. Notamos
que \(\tau(1_K)\star a = 1_Ka = a = a\star
\tau(1_K)\). Luego \(\tau(1_K) := 1_A\) actua como elemento
neutro de \(A\) para la operación \(\star\). Si comprobamos que \(A\) con
\((\star, 1_A)\) es un anillo, entonces tendremos que \(A\) es una
\(K\)-álgebra. Como la operación es asociativa por hipótesis y ya tenemos el
elemento neutro, solo nos quedaría comprobar la distributividad que la tenemos
por ser \(\star\) una aplicación bilineal.
