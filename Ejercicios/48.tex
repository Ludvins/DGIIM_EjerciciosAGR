\section{Ejercicio 48. *}
\emph{Calcular explícitamente una representacion real no trivial de grado 2 del grupo de permutaciones \(S_{3}\)}.\\

Definimos un espacio vectorial auxiliar de dimensión \(2\),
\[
  V = \Big\{ v \in \mathbb{R}^{3} \ : \ v = (a_{1},a_{2},a_{3}) \text{ con } a_{1}+a_{2}+a_{3} = 0 \Big\}
\]
Una base de \(V\) es \(\{v_{1}, v_{2}\}\) con \(v_{1} = (1,-1,0)\) y \(v_{2} = (0,1,-1)\). Consideramos entonces el morfismo
\[
  \pi: S_{3} \to GL(V)
\]
tal que dado \(p \in S_{3}\), \(\pi(p)\) aplica la permutación al vector de coordenadas, que es un morfismo por construcción:
\[
  \pi(p_{1}p_{2})(v) = p_{1}p_{2}v = \pi(p_{1})(p_{2}v) = (\pi(p_{1})\circ \pi(p_{2})) (v).
\]
\[
  \pi(p)(1) = 1
\]
Mostramos dicho morfismo sobre el espacio de matrices, \(GL(V) \in \mathcal{M}_{2}(\mathbb{R})\), por ejemplo tomando la permutación \((1,2,3)\).
\[
  \pi((1,2,3))(v_{1}) = (0,1,-1) = v_{2} \quad \pi((1,2,3))(v_{1}) = (-1,0,1) = -v_{2}-v_{1}\\
\]
Por tanto,
\[
  \pi((1,2,3)) = \begin{pmatrix} 0 & -1\\ -1 & 1 \\ \end{pmatrix}
\]
Definimos de esta forma el morfismo completo
\[
  \begin{aligned}
    \pi: S_{3} &\to GL(V)\\
    1 &\mapsto  \begin{pmatrix} 1 & 0\\ 0 & 1 \\ \end{pmatrix}\\
    (1,2,3) &\mapsto  \begin{pmatrix} 0 & -1\\ 1 & -1 \\ \end{pmatrix}\\
    (1,3,2) &\mapsto  \begin{pmatrix} -1 & 1\\ -1 & 0 \\ \end{pmatrix}\\
    (1,2) &\mapsto  \begin{pmatrix} -1 & 1\\ 0 & 1 \\ \end{pmatrix}\\
    (1,3) &\mapsto  \begin{pmatrix} 0 & -1\\ -1 & 0 \\ \end{pmatrix}\\
    (2,3) &\mapsto  \begin{pmatrix} 1 & 0\\ 1 & -1 \\ \end{pmatrix}\\
  \end{aligned}
\]
