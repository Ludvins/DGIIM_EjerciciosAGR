\section{Ejercicio 54. }\emph{ Sea \(H\) un subgrupo notmal de \(G\) y \(\pi:G \to G/H\) la proyección canónica. Demostrar }
\[
  (V,\rho) \text{ \emph{es una repr. irr. de} } G/H \iff (V,\rho \circ \pi) \text{ \emph{es una repr. irr. de} } G
\]\\

Es sencillo comprobar que
\[
  (V,\rho) \text{ es una representación } \iff (V, \rho \circ \pi ) \text{ es una representación }
\]
por ser \(\pi\) un epimorfismo de grupos.\\

Por otro lado, por el mismo motivo, tenemos que estos conjuntos de subespacios de \(V\) son iguales.
\[
   \{ W \ :  ((\rho \circ \pi)(g))(W) \subset W \ \forall g \in G\} = \{ W \ : \rho(g+H)(W) \subset W \ \forall g+H \in G/H\}
\]

Por lo tanto, queda demostrada la irreducibilidad en ambos sentidos.
