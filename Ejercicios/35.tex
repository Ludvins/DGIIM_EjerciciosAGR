\section{Ejercicio 35. ** Sin Terminar.} \textit{Sea \(V\) un \(K\)-espacio
  vectorial de dimensión finita \(n\) y \(T:V \to V\) una
  aplicación lineal. Diremos que un vector \(v \in V\) es cíclico para
  \(T\) si \(\{v, T(v), \dots, T^{n-1}(v)\}\) es una base de \(V\)
  como \(K\)-espacio vectorial. Demostrar que \(V\) admite un vector cíclico si,
  y sólo si, el polinomio mínimo de \(T\) tiene grado \(n\). ¿Cuál es entonces
  la longitud de \(V\) en tanto que \(K[X]\)-módulo.}\\


Comencemos viendo que si admite un vector cíclico entonces el polinomio mínimo
tiene grado \(n\). Sabemos que el polinomio mínimo satisface que \(m(T)=0\). Supongamos que el
grado de \(m\) es \(n' < n\),  entonces \(m(T)=0 \implies T^{n}\) es una
combinación lineal de \(\{Id,T,\dots,T^{n'-1}\}\) por lo tanto \(T^{n'}(v)\) se
puede escribir como combinación lineal de \(\{v,T(v),\dots,T^{n'-1}(v)\}\),
luego no puede existir un vector cíclico.\\



\noindent\makebox[\linewidth]{\rule{\paperwidth}{0.4pt}}

Veamos ahora que si el polinomio mínimo tiene grado \(n\), entonces existe un
vector cíclico. Notamos que si el grado del polinomio mínimo es \(n\), entonces
coincide con el polinomio característico. Sea \(N\) la matriz compañera de
ambos. Sea \(\{e_1,\dots,e_n\}\) la base usual de \(V\), sabemos que la matriz
compañera verifica que \(N^i e_1 = e_{i+1}\), luego, nuestro objetivo es
verificar que existe una matriz de cambio de base \(P\), tal que \(P^{-1}TP =
N\) visto \(T\) como matriz. En ese caso tendremos que sea \(v = P(e_1)\), el conjunto
\(\{v,\dots,T^{n-1}v\}\) sería una base de \(V\), luego \(v\) sería un vector
cíclico.\\

Veamos que dos matrices son similares si tienen el mismo polinomio mínimo, sean
\(A, B\) las matrices similares tales que \(A = S^{-1}BS\), sea \(m_A\) el
polinomio mínimo de \(A\), se tiene
\[
  0 = m_A(A) = m_A(S^{-1}BS) = S^{-1}m_A(B)S
\]
donde la última igualdad viene de evaluar el polinomio mínimo en \(S^{-1}BS\) y
sacar factor común de cada término. Con esto tenemos que \(m_A\) anula \(B\), y
de forma similar podemos ver que \(m_B\) anula a \(A\). Como los polinomios
mínimos son los mónicos de menor grado que anulan a su matriz, \(m_A = m_B\).\\

Veamos ahora que el polinomio mínimo se descompone en producto de factores
lineales, para ello consideremos una matriz con un único bloque de Jordan \(J\)
(una matriz y si forma canónica de Jordan son similares), el polinomio
característico es de la forma \((t-\lambda)^n\) con \(\lambda\) el valor propio
correspondiente, además como \((J - \lambda I)^k\neq 0 \ \forall k < n\), el
polinomio mínimo \(m = (t-\lambda)^n\). Consideremos ahora que \(J\) es la
matriz de Jordan formada por dos bloques sobre el mismo valor propio
\(\lambda\), cada bloque de dimensiones \(n_1\) y \(n_2\) respectivamente y
\(n_1 \geq n_2\). El
polinomio característico es \((t-\lambda)^{n_1+n_2}\), pero \((J-\lambda I)^{n_1}
= 0\), luego el polinomio mínimo es \((t-\lambda)^{n_1}\). Esto nos permite
concluir que el polinomio mínimo de una matriz de Jordan cualquiera es de la
forma

\[
  m = \prod(t - \lambda_i)^{r_i}
\]
Donde \(\lambda_i\) son los valores propios y \(r_i\) el tamaño del mayor bloque correspondiente.\\

Además, en caso de que el polinomio mínimo
tenga grado \(n\), se tiene que \(n = r_1 + \dots + r_m\), luego solo existe un
bloque de Jordan por cada valor propio, entonces, \(A\) y la matriz compañera de
su polinomio característico \(C\), tienen la misma forma canónica de Jordan, un
bloque de tamaño \(r_i\) para cada valor propio \(\lambda_i\), luego ambas son
similares a la misma matriz de Jordan, luego son similares entre si.

