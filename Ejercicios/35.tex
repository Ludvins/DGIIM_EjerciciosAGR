\section{Ejercicio 35. **} \textit{Sea \(V\) un \(K\)-espacio
  vectorial de dimensión finita \(n\) y \(T:V \to V\) una
  aplicación lineal. Diremos que un vector \(v \in V\) es cíclico para
  \(T\) si \(\{v, T(v), \dots, T^{n-1}(v)\}\) es una base de \(V\)
  como \(K\)-espacio vectorial. Demostrar que \(V\) admite un vector cíclico si,
  y sólo si, el polinomio mínimo de \(T\) tiene grado \(n\). ¿Cuál es entonces
  la longitud de \(V\) en tanto que \(K[X]\)-módulo?.}\\


Comencemos viendo que si admite un vector cíclico entonces el polinomio mínimo
tiene grado \(n\). Sabemos que el polinomio mínimo satisface que \(m(T)=0\). Supongamos que el
grado de \(m\) es \(n' < n\),  entonces \(m(T)=0 \implies T^{n'}\) es una
combinación lineal de \(\{Id,T,\dots,T^{n'-1}\}\) por lo tanto \(T^{n'}(v)\) se
puede escribir como combinación lineal de \(\{v,T(v),\dots,T^{n'-1}(v)\}\),
luego no puede existir un vector cíclico.\\

Supongamos ahora que el polinomio mínimo \(m\) tiene grado \(n\). Sea
\[
  m = \prod_{i=0}^k m_i^{n_i}
\]
la descomposición de \(m\) en factores irreducibles y consideremos
\[
  V_i = Ker(m_i^{n_i}(T)) = \{v \in V \ | \ m_i^{n_i}(T)(v) = 0\} \ \forall i = 1,\dots,k
\]
Utilizando un proceso inductivo en la primera proposición del ejercicio 26, podemos ver
que \(V = \oplus_{i=0}^k V_i\), además, es sencillo comprobar que son submódulos de \(V\) puesto que \(T\) y \(m_i^{n_i}(T)\)
conmutan, luego \(m_i^{n_i}(T)(T(v)) = T(m_i^{n_i}(T)(v)) = 0\).\\

Notemos que \(m_i^{n_i - 1}(T)_{|V_i} \neq 0\) en \(\forall i=1,\dots,k\), ya que en caso
contrario, lo multiplicaríamos por el resto de factores irreducibles y
tendríamos un polinomio que anula todo \(V\) de grado menor \(n\). Entonces existe \(v_i \in V_i\backslash Ker(m_i^{n_i-1}(T)) \ \forall i=0,\dots,k\).\\

Si algún \(v \in V_i\) se anulara en un \(m_j(T)\) con \(i \neq j\),
entonces el polinomio mínimo asociado al submódulo \(\langle v \rangle\) dividiría a \(m_j\) y
\(m_i^{n_i}\), que como son primos relativos, lo forzarían a ser 1, luego \(v = 0\).

Consideramos entonces \(x = \sum_i v_i\), tenemos que ningún divisor de \(m\)
evaluado en \(T\) anula a \(x\). Supongamos entonces que un polinomio \(p\) de
grado menor que \(n\) que no es un divisor de \(m\) y cumple \(p(T)(x) = 0\),
entonces por ser primos relativos, existen \(u,v\) tales que \(1 = up + vm
\implies x = u(T)p(T)x + v(T)m(T)x = 0 \), lo cual no puede suceder por ser
\(v_i \neq 0 \ \forall i=0,\dots,k\), lo cual no puede suceder por ser \(v_i \neq 0 \
\forall i\). \\

Entonces \(x\) es un vector que solo se anula
por \(m\) y múltiplos suyos. Es decir, \(\{x, T(x), \dots, T^{n-1}(x)\}\)
son linealmente independientes, luego forman una base de \(V\) como espacio
vectorial.\\
