\section{Ejercicio 35. ** Sin Terminar.} \textit{Sea \(V\) un \(K\)-espacio
  vectorial de dimensión finita \(n\) y \(T:V \to V\) una
  aplicación lineal. Diremos que un vector \(v \in V\) es cíclico para
  \(T\) si \(\{v, T(v), \dots, T^{n-1}(v)\}\) es una base de \(V\)
  como \(K\)-espacio vectorial. Demostrar que \(V\) admite un vector cíclico si,
  y sólo si, el polinomio mínimo de \(T\) tiene grado \(n\). ¿Cuál es entonces
  la longitud de \(V\) en tanto que \(K[X]\)-módulo?.}\\


Comencemos viendo que si admite un vector cíclico entonces el polinomio mínimo
tiene grado \(n\). Sabemos que el polinomio mínimo satisface que \(m(T)=0\). Supongamos que el
grado de \(m\) es \(n' < n\),  entonces \(m(T)=0 \implies T^{n'}\) es una
combinación lineal de \(\{Id,T,\dots,T^{n'-1}\}\) por lo tanto \(T^{n'}(v)\) se
puede escribir como combinación lineal de \(\{v,T(v),\dots,T^{n'-1}(v)\}\),
luego no puede existir un vector cíclico.\\

Supongamos ahora que el polinomio mínimo \(m\) tiene grado \(n\). Sea
\[
  m = \prod_i m_i^{n_i}
\]
la descomposición de \(m\) en factores irreducibles y consideremos
\[
  V_i = Ker(m_i(T)) = \{v \in V \ | \ m_i^{n_i}(T)(v) = 0\}
\]
Utilizando un proceso inductivo en la proposición del ejercicio 26, podemos ver
que \(V = \oplus_i V_i\), además, es sencillo comprobar que son submódulos de \(V\) puesto que \(T\) y \(m_i^{n_i}(T)\)
conmutan, luego \(m_i^{n_i}(T)(T(v)) = T(m_i^{n_i}(T)(v)) = 0\).
Notemos que \(m_i^{n_i - 1}(T) \neq 0\) en \(V_i \ \forall i\), ya que en caso
contrario, lo multiplicaríamos por el resto de factores irreducibles y
tendríamos un polinomio que anula todo \(V\) de grado menor \(n\).\\

Entonces existe \(v_i \not \in Ker(m_i^{n_i-1}(T)) \ \forall i\).\\

Si alguno de estos \(v_i\) se anulara en un \(m_j(T)\) con \(i \neq j\),
entonces el polinomio mínimo de \(\langle v_i \rangle\) divide a \(m_j\) y
\(m_i^{n_i}\), que como son primos relativos resulta ser 1, luego \(v_i = 0\).

Consideramos entonces \(x = \oplus_i v_i\), tenemos que ningún divisor de \(m\)
evaluado en \(T\) anula a \(x\). Entonces \(x\) es un vector que solo se anula
por \(m\) y múltiplos suyos, luego polinomio de menor grado evaluado en \(T\) y
en \(x\) no resulta 0. Es decir, \(\{x, T(x), \dots, T^{n-1}(x)\}\)
son linealmente independientes, luego forman una base de \(V\) como espacio
vectorial.
