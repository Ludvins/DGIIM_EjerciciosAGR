\section{Ejercicio 35. **} \textit{Sea \(V\) un \(K\)-espacio
  vectorial de dimensión finita \(n\) y \(T:V \to V\) una
  aplicación lineal. Diremos que un vector \(v \in V\) es cíclico para
  \(\mathcal{T}\) si \(\{v, T(v), \dots, T^{n-1}(v)\}\) es una base de \(V\)
  como \(K\)-espacio vectorial. Demostrar que \(V\) admite un vector cíclico si,
  y sólo si, el polinomio mínimo de \(T\) tiene grado \(n\). ¿Cuál es entonces
  la longitud de \(V\) en tanto que \(K[X]\)-módulo.}\\


Comencemos viendo que si admite un vector cíclico entonces el polinomio mínimo
tiene grado \(n\). Sabemos que el polinomio mínimo satisface que \(m(T)=0\). Supongamos que el
grado de \(m\) es \(n' < n\),  entonces \(m(T)=0 \implies T^{n}\) es una
combinación lineal de \(\{Id,T,\dots,T^{n'-1}\}\) por lo tanto \(T^{n'}(v)\) se
puede escribir como combinación lineal de \(\{v,T(v),\dots,T^{n'-1}(v)\}\),
luego no puede existir un vector cíclico.\\

Veamos ahora que si el polinomio mínimo tiene grado \(n\), entonces existe un
vector cíclico. Notamos que si el grado del polinomio mínimo es \(n\), entonces
coincide con el polinomio característico. Sea \(\bar{N}\) la matriz compañera de
ambos y \(\{\alpha_1, \dots, \alpha_n\}\) una base de \(V\).

\textbf{TODO. Terminar}
