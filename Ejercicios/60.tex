\section{Ejercicio 60 *.} \emph{Calcula razonadamente la tabla de caracteres del grupo diédrico \(D_{4}\)}.\\

Las clases de conjugación de \(D_{4} = \langle r,s \rangle\) son
\[
  \{1\}, \{r^{2}\}, \{s, r^{2}s\}, \{r, r^{3}\}, \{rs, r^{3}s\}
\]
Luego tenemos 5 caracteres complejos irreducibles. Sea \(\rho\) una representación compleja irreducible de grado 1, tenemos que \(\rho(r)^{4} = \rho(r^{4}) = 1\), luego \(\rho(r)\) es una raiz cuarta de la unidad \(\{1, -1, i, -i\}\). Por el mismo motivo, \(\rho(s)\) es una raiz cuadrada de la unidad \(\{1, -1\}\).\\

Por otro lado, como \(\rho(r^{-1}) = \rho(srs) = \rho(s)\rho(r)\rho(s) = \rho(r)\), luego \(\rho(r)^{-1} = \rho(r) \implies \rho(r)^{2} = 1 \implies \rho(r) \in \{-1, 1\}\).\\

Por lo tanto, \(\rho(\pi(r)) = \pm 1 = \rho(\pi(s))\). Con esto tenemos 4 posibles representacions dependiendo de lsa configuraciones de \(\rho(r)\) y \(\rho(s)\).

\begin{figure}[H]
  \centering
  \begin{tabular}{c|ccccc}
    & \(1\) & \(1\) & \(2\) & \(2\) & \(2\) \\
      \(D_{4}\)  & \(1\)  & \(r^{2}\)  & \(\{s, r^{2}s\}\) & \(\{r,r^{3}\}\)  & \(\{rs, r^{3}s\}\) \\ \hline
      \(\mathcal{X}_{1}\) & \(1\)   &  \(1\)    &     \(1\)      &     \(1\)  & \(1\)       \\
      \(\mathcal{X}_{2}\) & \(1\)   &  \(1\)    &     \(1\)      &     \(-1\)  & \(-1\)       \\
      \(\mathcal{X}_{3}\) & \(1\)   &  \(1\)    &     \(-1\)      &     \(1\)  & \(-1\)       \\
      \(\mathcal{X}_{4}\) & \(1\)   &  \(1\)    &     \(-1\)      &     \(-1\)  & \(1\)       \\
      \(\mathcal{X}_{5}\) & \(2\)   &  \(x\)    &     \(y\)      &     \(z\)  & \(t\)       \\
    \end{tabular}
\end{figure}

Donde \(\mathcal{X}_{5,1} = 2\) ya que \(8 = 1+1+1+1+2^{2}\). Sacamos el resto de valores de resolver las siguientes ecuaciones de ortogonalidad
\[
  \begin{aligned}
    2 + x + y + z + t &= 0\\
    2 + x + y - z - t &= 0\\
    2 + x - y + z - t &= 0\\
    2 + x - y - z + t &= 0\\
  \end{aligned}
\]
De donde sacamos que \(x = -2 \), \(y = 0\), \(z = 0\) y \(t = 0\). Quedando la tabla
\begin{figure}[H]
  \centering
  \begin{tabular}{c|ccccc}
    & \(1\) & \(1\) & \(2\) & \(2\) & \(2\) \\
      \(D_{4}\)  & \(1\)  & \(r^{2}\)  & \(\{s, r^{2}s\}\) & \(\{r,r^{3}\}\)  & \(\{rs, r^{3}s\}\) \\ \hline
      \(\mathcal{X}_{1}\) & \(1\)   &  \(1\)    &     \(1\)      &     \(1\)  & \(1\)       \\
      \(\mathcal{X}_{2}\) & \(1\)   &  \(1\)    &     \(1\)      &     \(-1\)  & \(-1\)       \\
      \(\mathcal{X}_{3}\) & \(1\)   &  \(1\)    &     \(-1\)      &     \(1\)  & \(-1\)       \\
      \(\mathcal{X}_{4}\) & \(1\)   &  \(1\)    &     \(-1\)      &     \(-1\)  & \(1\)       \\
      \(\mathcal{X}_{5}\) & \(2\)   &  \(-2\)    &     \(0\)      &     \(0\)  & \(0\)       \\
    \end{tabular}
\end{figure}
