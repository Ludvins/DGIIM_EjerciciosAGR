\section{Ejercicio 51.} \emph{Calcular todos los subespacios invariantes para la representación de \(S_{3}\) dada en el ejercicio 48.}\\

Veamos que no existe ningún subespacio propio invariante, para ello, notamos que \(V\) tiene dimensión 2, luego un subespacio propio tendrá dimensión 1. Es decir, \(\exists a,b\in \mathbb{R}\) tal que \(W = \langle (a,b) \rangle\) en coordenadas con respecto a la base \(\{v_{1}, v_{2}\}\).\\

Consideramos ahora la permutación \((1 \ 3)\), tenemos que
\[
  \pi((1 \ 3))(W) \subset W \implies \pi((1 \ 3))(a, b) = (-b, -a) \in \langle (a, b) \rangle
\]
Pero \((-b,-a)\perp (a,b)\), por lo tanto, no existen subespacios propios invariantes.
